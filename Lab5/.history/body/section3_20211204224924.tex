%! TEX program = xelatex
%! TEX root = ../root.tex

\section{实验原理}
\subsection{Datapath}
为了支持多周期操作本实验对数据通路进行了修改,想要成功完成此次实验,对Datapath的熟练掌握自然是必要的。
下面就是本次实验所参照的Datapath

\begin{figure}[H] %H为当前位置,!htb为忽略美学标准,htbp为浮动图形
	\centering %图片居中
	\includegraphics[width=1.0\textwidth]{figs/DataPath.png} %插入图片,[]中设置图片大小,{}中是图片文件名
	\caption{Datapath} %最终文档中希望显示的图片标题
	\label{Fig.1} %用于文内引用的标签
\end{figure}

根据此图,我们就能够将数据通路部分完成,完成的RV32Core.v代码如下所示:
\begin{lstlisting}[language = {verilog}]
`timescale 1ns / 1ps

module  RV32core(
		input debug_en,  // debug enable
		input debug_step,  // debug step clock
		input [6:0] debug_addr,  // debug address
		output[31:0] debug_data,  // debug data
		input clk,  // main clock
		input rst,  // synchronous reset
		input interrupter  // interrupt source, for future use
	);

	wire debug_clk;
	wire[31:0] debug_regs;
	reg[31:0] Test_signal;
	assign debug_data = debug_addr[5] ? Test_signal : debug_regs;

	debug_clk clock(.clk(clk),.debug_en(debug_en),.debug_step(debug_step),.debug_clk(debug_clk));

	wire reg_IF_EN, reg_ID_EN, reg_ID_flush, FU_ALU_EN, FU_mem_EN, FU_mul_EN, FU_div_EN, FU_jump_EN;
	wire RegWrite_ctrl, ALUSrcA_ctrl, ALUSrcB_ctrl, mem_w_ctrl, branch_ctrl;
	wire[2:0] ImmSel_ctrl, DatatoReg_ctrl;
	wire[3:0] ALUControl_ctrl, Jump_ctrl;
	wire[4:0] rd_ctrl;


	wire [31:0] PC_IF, next_PC_IF, PC_4_IF, inst_IF;

	wire valid_ID;
	wire[31:0]inst_ID, PC_ID, Imm_out_ID, rs1_data_ID, rs2_data_ID, ALUA_ID, ALUB_ID;

	wire FU_ALU_finish, FU_mem_finish, FU_mul_finish, FU_div_finish, FU_jump_finish, cmp_res_FU;
	wire[31:0]ALUout_FU, mem_data_FU, mulres_FU, divres_FU, PC_jump_FU, PC_wb_FU;

	wire[31:0]ALUout_WB, mem_data_WB, mulres_WB, divres_WB, PC_wb_WB, wt_data_WB;


	// IF
	REG32 REG_PC(.clk(debug_clk),.rst(rst),.CE(reg_IF_EN),.D(next_PC_IF),.Q(PC_IF));
	
	add_32 add_IF(.a(PC_IF),.b(32'd4),.c(PC_4_IF));

	MUX2T1_32 mux_IF(.I0(PC_4_IF),.I1(PC_jump_FU),.s(branch_ctrl),.o(next_PC_IF));

	ROM_D inst_rom(.a(PC_IF[8:2]),.spo(inst_IF));


	//Issue
	REG_ID reg_ID(.clk(debug_clk),.rst(rst),.EN(reg_ID_EN),
		.flush(reg_ID_flush),.PCOUT(PC_IF),.IR(inst_IF),

		.IR_ID(inst_ID),.PCurrent_ID(PC_ID),.valid(valid_ID));
	
	CtrlUnit ctrl(.clk(debug_clk),.rst(rst),.inst(inst_ID),.valid_ID(valid_ID),
		.ALU_done(FU_ALU_finish),.MEM_done(FU_mem_finish),.MUL_done(FU_mul_finish),
		.DIV_done(FU_div_finish),.JUMP_done(FU_jump_finish),.cmp_res_FU(cmp_res_FU),

		.reg_IF_en(reg_IF_EN),.branch_ctrl(branch_ctrl),.reg_ID_en(reg_ID_EN),
		.reg_ID_flush(reg_ID_flush),.ImmSel(ImmSel_ctrl),.ALU_en(FU_ALU_EN),
		.MEM_en(FU_mem_EN),.MUL_en(FU_mul_EN),.DIV_en(FU_div_EN),.JUMP_en(FU_jump_EN),
		.JUMP_op(Jump_ctrl),.ALU_op(ALUControl_ctrl),.MEM_we(mem_w_ctrl),
		.ALUSrcA(ALUSrcA_ctrl),.ALUSrcB(ALUSrcB_ctrl),
		.write_sel(DatatoReg_ctrl),.reg_write(RegWrite_ctrl),.rd_ctrl(rd_ctrl));

	ImmGen imm_gen(.ImmSel(ImmSel_ctrl), .inst_field(inst_ID), .Imm_out(Imm_out_ID));            //to fill sth.in

	Regs register(.clk(debug_clk),.rst(rst),
		.R_addr_A(inst_ID[19:15]),.rdata_A(rs1_data_ID),
		.R_addr_B(inst_ID[24:20]),.rdata_B(rs2_data_ID),
		.L_S(RegWrite_ctrl),.Wt_addr(rd_ctrl),.Wt_data(wt_data_WB),
		.Debug_addr(debug_addr[4:0]),.Debug_regs(debug_regs));

	MUX2T1_32 mux_imm_ALU_ID_A(.I0(rs1_data_ID), .I1(PC_ID), .s(ALUSrcA_ctrl), .o(ALUA_ID));            //to fill sth.in

	MUX2T1_32 mux_imm_ALU_ID_B(.I0(rs2_data_ID), .I1(PC_ID), .s(ALUSrcB_ctrl), .o(ALUB_ID));            //to fill sth.in


	// FU
	FU_ALU alu(.clk(debug_clk),.EN(FU_ALU_EN),.finish(FU_ALU_finish),
		.ALUControl(ALUControl_ctrl),.ALUA(ALUA_ID),.ALUB(ALUB_ID),.res(ALUout_FU),
		.zero(),.overflow());

	FU_mem mem(.clk(debug_clk),.EN(FU_mem_EN),.finish(FU_mem_finish),
		.mem_w(mem_w_ctrl),.bhw(inst_ID[14:12]),.rs1_data(rs1_data_ID),.rs2_data(rs2_data_ID),
		.imm(Imm_out_ID),.mem_data(mem_data_FU));

	FU_mul mu(.clk(debug_clk),.EN(FU_mul_EN),.finish(FU_mul_finish),
		.A(rs1_data_ID),.B(rs2_data_ID),.res(mulres_FU));

	FU_div du(.clk(debug_clk),.EN(FU_div_EN),.finish(FU_div_finish),
		.A(rs1_data_ID),.B(rs2_data_ID),.res(divres_FU));

	FU_jump ju(.clk(debug_clk),.EN(FU_jump_EN),.finish(FU_jump_finish),
		.JALR(Jump_ctrl[3]),.cmp_ctrl(Jump_ctrl[2:0]),.rs1_data(rs1_data_ID),.rs2_data(rs2_data_ID),
		.imm(Imm_out_ID),.PC(PC_ID),.PC_jump(PC_jump_FU),.PC_wb(PC_wb_FU),.cmp_res(cmp_res_FU));


	// WB
	REG32 reg_WB_ALU(.clk(debug_clk),.rst(rst),.CE(FU_ALU_finish),.D(ALUout_FU),.Q(ALUout_WB));

	REG32 reg_WB_mem(.clk(debug_clk),.rst(rst),.CE(FU_mem_finish),.D(mem_data_FU),.Q(mem_data_WB));

	REG32 reg_WB_mul(.clk(debug_clk),.rst(rst),.CE(FU_mul_finish),.D(mulres_FU),.Q(mulres_WB));

	REG32 reg_WB_div(.clk(debug_clk),.rst(rst),.CE(FU_div_finish),.D(divres_FU),.Q(divres_WB));
	
	REG32 reg_WB_jump(.clk(debug_clk),.rst(rst),.CE(FU_jump_finish),.D(PC_wb_FU),.Q(PC_wb_WB));

	MUX8T1_32 mux_DtR(.s(DatatoReg_ctrl), .I0(32'd0), .I1(ALUout_WB), .I2(mem_data_WB), .I3(mulres_WB),
	.I4(divres_WB), .I5(PC_wb_WB), .I6(32'd0), .I7(32'd0), .o(wt_data_WB));  //to fill sth.in


	always @* begin
		case (debug_addr[4:0])
			0:  Test_signal = PC_IF;
			1:  Test_signal = inst_IF;
			2:  Test_signal = PC_ID;  
			3:  Test_signal = inst_ID;

			4:  Test_signal = inst_ID[19:15];
			5:  Test_signal = rs1_data_ID;
			6:  Test_signal = inst_ID[24:20];
			7:  Test_signal = rs2_data_ID;

			8:  Test_signal = ImmSel_ctrl;
			9:  Test_signal = Imm_out_ID;
			10: Test_signal = ALUout_FU;
			11: Test_signal = reg_IF_EN;

			12: Test_signal = {15'b0, FU_ALU_EN, 15'b0, FU_ALU_finish};
			13: Test_signal = ALUControl_ctrl;
			14: Test_signal = ALUA_ID;
			15: Test_signal = ALUB_ID;

			16: Test_signal = {15'b0, FU_mem_EN, 15'b0, FU_mem_finish};
			17: Test_signal = mem_w_ctrl;
			18: Test_signal = inst_ID[14:12];
			19: Test_signal = mem_data_FU;

			20: Test_signal = {15'b0, FU_mul_EN, 15'b0, FU_mul_finish};
			21: Test_signal = mulres_FU;
			22: Test_signal = {15'b0, FU_div_EN, 15'b0, FU_div_finish};
			23: Test_signal = divres_FU;

			24: Test_signal = {15'b0, FU_jump_EN, 15'b0, FU_jump_finish};
			25: Test_signal = Jump_ctrl;
			26: Test_signal = PC_jump_FU;
			27: Test_signal = PC_wb_FU;

			28: Test_signal = RegWrite_ctrl;
			29: Test_signal = rd_ctrl;
			30: Test_signal = DatatoReg_ctrl;
			31: Test_signal = wt_data_WB;
			
			default: Test_signal = 32'hAA55_AA55;
		endcase
	end

endmodule
\end{lstlisting}

\subsection{Data Hazard}
但流水线支持多周期操作后,势必会更改Data Hazard的情形和解决方式,为了解决这一问题,
此实验中简单的使用了FU阶段的各操作的finish信号进行Data Hazard的resolve,只有当一条指令的FU阶段结束之后才允许
进入WB阶段,当然仅仅只控制当前执行指令停顿是远远不够的,此次实验中还将FU的结束信号传入了Ctrl Unit中,作为Data Hazard的检测信号,
当FU阶段未结束时,Ctrl Unit中的FU\_in\_use信号会升起,并且会使得后续指令同样等待FU阶段结束。\\
具体例子如下所示:

\begin{figure}[H] %H为当前位置,!htb为忽略美学标准,htbp为浮动图形
	\centering %图片居中
	\includegraphics[width=1.0\textwidth]{figs/DataHazard.png} %插入图片,[]中设置图片大小,{}中是图片文件名
	\caption{DataHazard} %最终文档中希望显示的图片标题
	\label{Fig.2} %用于文内引用的标签
\end{figure}

\subsection{Control Hazard}
Ctrl Unit中不仅进行Data Hazard检测与解决,还实现了Control Unit的resolve操作,此部分的关键点就在于:在本次实验中
有关跳转操作的地址计算与跳转条件判断均是在FU阶段计算完成的,结合FU阶段的计算结果和FU结束的信号,可以进行Control Hazard的解决。\\

本实验还实现了Predict Not Taken策略,实现过程同样也是在Ctrl Unit中完成,默认认为跳转不会执行,在跳转指令进入FU阶段前正常取指与执行
但FU阶段结束,跳转需要被执行时,通过Ctrl Unit中的reg\_ID\_flush信号升起,来flush掉ID段寄存器,并且将跳转到的指令取入IF段寄存器。

总得来说,过程如下图所示:

\begin{figure}[H] %H为当前位置,!htb为忽略美学标准,htbp为浮动图形
	\centering %图片居中
	\includegraphics[width=1.0\textwidth]{figs/Branch.png} %插入图片,[]中设置图片大小,{}中是图片文件名
	\caption{Control Hazard} %最终文档中希望显示的图片标题
	\label{Fig.3} %用于文内引用的标签
\end{figure}

根据上述设计思想,Ctrl Unit设计如下:
\begin{lstlisting}[language=Verilog]
`timescale 1ns / 1ps

module CtrlUnit(
	input clk,
	input rst,

	input[31:0] inst,
	input valid_ID,
	
	input ALU_done,
	input MEM_done,
	input MUL_done,
	input DIV_done,
	input JUMP_done,
	input cmp_res_FU,

	// IF
	output reg_IF_en, branch_ctrl,

	// ID
	output reg_ID_en, reg_ID_flush,
	output[2:0] ImmSel,
	output ALU_en, MEM_en, MUL_en, DIV_en, JUMP_en,
	
	// FU
	output[3:0] JUMP_op,
	output[3:0] ALU_op,
	output ALUSrcA,
	output ALUSrcB,
	output MEM_we,
	
	// WB
	output reg[2:0] write_sel,
	output reg[4:0] rd_ctrl,
	output reg reg_write
);
		// instruction field
	wire[6:0] funct7 = inst[31:25];
	wire[2:0] funct3 = inst[14:12];
	wire[6:0] opcode = inst[6:0];
	wire[4:0] rd = inst[11:7];
	wire[4:0] rs1 = inst[19:15];
	wire[4:0] rs2 = inst[24:20];

	// type specification
	wire Rop = opcode == 7'b0110011;
	wire Iop = opcode == 7'b0010011;
	wire Bop = opcode == 7'b1100011;
	wire Lop = opcode == 7'b0000011;
	wire Sop = opcode == 7'b0100011;

	wire funct7_0  = funct7 == 7'h0;
	wire funct7_1  = funct7 == 7'h1;
	wire funct7_32 = funct7 == 7'h20;

	wire funct3_0 = funct3 == 3'h0;
	wire funct3_1 = funct3 == 3'h1;
	wire funct3_2 = funct3 == 3'h2;
	wire funct3_3 = funct3 == 3'h3;
	wire funct3_4 = funct3 == 3'h4;
	wire funct3_5 = funct3 == 3'h5;
	wire funct3_6 = funct3 == 3'h6;
	wire funct3_7 = funct3 == 3'h7;

	wire ADD  = Rop & funct3_0 & funct7_0;
	wire SUB  = Rop & funct3_0 & funct7_32;
	wire SLL  = Rop & funct3_1 & funct7_0;
	wire SLT  = Rop & funct3_2 & funct7_0;
	wire SLTU = Rop & funct3_3 & funct7_0;
	wire XOR  = Rop & funct3_4 & funct7_0;
	wire SRL  = Rop & funct3_5 & funct7_0;
	wire SRA  = Rop & funct3_5 & funct7_32;
	wire OR   = Rop & funct3_6 & funct7_0;
	wire AND  = Rop & funct3_7 & funct7_0;

	wire MUL    = Rop & funct3_0 & funct7_1;
	wire MULH   = Rop & funct3_1 & funct7_1;
	wire MULHSU = Rop & funct3_2 & funct7_1;
	wire MULHU  = Rop & funct3_3 & funct7_1;
	wire DIV    = Rop & funct3_4 & funct7_1;
	wire DIVU   = Rop & funct3_5 & funct7_1;
	wire REM    = Rop & funct3_6 & funct7_1;
	wire REMU    = Rop & funct3_7 & funct7_1;

	wire ADDI  = Iop & funct3_0;	
	wire SLTI  = Iop & funct3_2;
	wire SLTIU = Iop & funct3_3;
	wire XORI  = Iop & funct3_4;
	wire ORI   = Iop & funct3_6;
	wire ANDI  = Iop & funct3_7;
	wire SLLI  = Iop & funct3_1 & funct7_0;
	wire SRLI  = Iop & funct3_5 & funct7_0;
	wire SRAI  = Iop & funct3_5 & funct7_32;

	wire BEQ = Bop & funct3_0;
	wire BNE = Bop & funct3_1;
	wire BLT = Bop & funct3_4;
	wire BGE = Bop & funct3_5;
	wire BLTU = Bop & funct3_6;
	wire BGEU = Bop & funct3_7;

	wire LB =  Lop & funct3_0;
	wire LH =  Lop & funct3_1;
	wire LW =  Lop & funct3_2;
	wire LBU = Lop & funct3_4;
	wire LHU = Lop & funct3_5;

	wire SB = Sop & funct3_0;
	wire SH = Sop & funct3_1;
	wire SW = Sop & funct3_2;

	wire LUI   = opcode == 7'b0110111;
	wire AUIPC = opcode == 7'b0010111;

	wire JAL  =  opcode == 7'b1101111;
	wire JALR = (opcode == 7'b1100111) && funct3_0;

	wire R_valid = AND | OR | ADD | XOR | SLL | SRL | SRA | SUB | SLT | SLTU 
		| MUL | MULH | MULHSU | MULHU | DIV | DIVU | REM | REMU;
	wire I_valid = ANDI | ORI | ADDI | XORI | SLLI | SRLI | SRAI | SLTI | SLTIU;
	wire B_valid = BEQ | BNE | BLT | BGE | BLTU | BGEU;
	wire L_valid = LW | LH | LB | LHU | LBU;
	wire S_valid = SW | SH | SB;

	wire use_ALU = AND | OR | ADD | XOR | SLL | SRL | SRA | SUB | SLT | SLTU
		| I_valid | LUI | AUIPC;
	wire use_MEM = L_valid | S_valid;
	wire use_MUL = MUL | MULH | MULHSU | MULHU;
	wire use_DIV = DIV | DIVU | REM | REMU;
	wire use_JUMP = B_valid | JAL | JALR;

	wire[2:0] use_FU =  {3{use_ALU}}  & 3'd1 |
						{3{use_MEM}}  & 3'd2 |
						{3{use_MUL}}  & 3'd3 |
						{3{use_DIV}}  & 3'd4 |
						{3{use_JUMP}} & 3'd5 ;

	reg FU_in_use, to_writeback, B_in_FU, J_in_FU;
	initial begin
		FU_in_use = 0;
		to_writeback = 0;
		B_in_FU = 0;
		J_in_FU = 0;
		write_sel = 0;
		rd_ctrl = 0;
		reg_write = 0;
	end

	always @(posedge clk or posedge rst) begin
		if(rst) begin
			FU_in_use <= 0;
			reg_write <= 0;
			B_in_FU <= 0;
			J_in_FU <= 0;
		end
		else if(~FU_in_use) begin
			reg_write <= 0;
			if(valid_ID) begin
				FU_in_use <= 1;
				rd_ctrl <= inst[11:7];
				write_sel <= use_FU;
				to_writeback <= R_valid | I_valid | JAL | JALR | L_valid | LUI | AUIPC;
				B_in_FU <= B_valid;
				J_in_FU <= JAL | JALR;
			end
			else begin
				B_in_FU <= 0;
				J_in_FU <= 0;
			end
		end
		else if(ALU_done | MEM_done | MUL_done | DIV_done | JUMP_done) begin
			FU_in_use <= 0;
			reg_write <= to_writeback;
			B_in_FU <= 0;
			J_in_FU <= 0;
		end
		else begin
			reg_write <= 0;
			B_in_FU <= 0;
			J_in_FU <= 0;
		end
	end


	assign reg_IF_en = ~FU_in_use | branch_ctrl;

	assign reg_ID_en = reg_IF_en;

	assign branch_ctrl = (B_in_FU & cmp_res_FU) |  J_in_FU;

	assign reg_ID_flush = branch_ctrl;

	localparam Imm_type_I = 3'b001;
	localparam Imm_type_B = 3'b010;
	localparam Imm_type_J = 3'b011;
	localparam Imm_type_S = 3'b100;
	localparam Imm_type_U = 3'b101;
	assign ImmSel = {3{JALR | L_valid | I_valid}} & Imm_type_I |
					{3{B_valid}}                  & Imm_type_B |
					{3{JAL}}                      & Imm_type_J |
					{3{S_valid}}                  & Imm_type_S |
					{3{LUI | AUIPC}}              & Imm_type_U ;
	
	assign ALU_en = reg_IF_en & use_ALU & valid_ID;
	assign MEM_en = reg_IF_en & use_MEM & valid_ID;
	assign MUL_en = reg_IF_en & use_MUL & valid_ID;
	assign DIV_en = reg_IF_en & use_DIV & valid_ID;
	assign JUMP_en = reg_IF_en & use_JUMP & valid_ID;

	localparam JUMP_BEQ  = 4'b0_001;
	localparam JUMP_BNE  = 4'b0_010;
	localparam JUMP_BLT  = 4'b0_011;
	localparam JUMP_BGE  = 4'b0_100;
	localparam JUMP_BLTU = 4'b0_101;
	localparam JUMP_BGEU = 4'b0_110;
	localparam JUMP_JAL  = 4'b0_000;
	localparam JUMP_JALR = 4'b1_000;
	assign JUMP_op ={4{BEQ}}  & JUMP_BEQ  |
					{4{BNE}}  & JUMP_BNE  |
					{4{BLT}}  & JUMP_BLT  |
					{4{BGE}}  & JUMP_BGE  |
					{4{BLTU}} & JUMP_BLTU |
					{4{BGEU}} & JUMP_BGEU |
					{4{JAL}}  & JUMP_JAL  |
					{4{JALR}} & JUMP_JALR ;
	
	localparam ALU_ADD  = 4'b0001;
	localparam ALU_SUB  = 4'b0010;
	localparam ALU_AND  = 4'b0011;
	localparam ALU_OR   = 4'b0100;
	localparam ALU_XOR  = 4'b0101;
	localparam ALU_SLL  = 4'b0110;
	localparam ALU_SRL  = 4'b0111;
	localparam ALU_SLT  = 4'b1000;
	localparam ALU_SLTU = 4'b1001;
	localparam ALU_SRA  = 4'b1010;
	localparam ALU_Ap4  = 4'b1011;
	localparam ALU_Bout = 4'b1100;
	assign ALU_op = {4{ADD | ADDI | AUIPC}} & ALU_ADD  |
					{4{SUB}}                & ALU_SUB  |
					{4{AND | ANDI}}         & ALU_AND  |
					{4{OR | ORI}}           & ALU_OR   |
					{4{XOR | XORI}}         & ALU_XOR  |
					{4{SLL | SLLI}}         & ALU_SLL  |
					{4{SRL | SRLI}}         & ALU_SRL  |
					{4{SLT | SLTI}}         & ALU_SLT  |
					{4{SLTU | SLTIU}}       & ALU_SLTU |
					{4{SRA | SRAI}}         & ALU_SRA  |
					{4{LUI}}                & ALU_Bout ;

	assign ALUSrcA = AUIPC;

	assign ALUSrcB = I_valid | LUI | AUIPC;

	assign MEM_we = S_valid;
endmodule
\end{lstlisting}

\subsection{FU阶段}
\subparagraph{FU\_ALU} FU\_ALU执行的是ALU操作,由两个时钟周期完成,基本操作与ALU操作相同,但通过state和一系列reg寄存器来控制结束周期。具体设计如下:

\begin{lstlisting}[language = {verilog}]
`timescale 1ns / 1ps

module FU_ALU(
	input clk, EN,
	input[3:0] ALUControl,
	input[31:0] ALUA, ALUB,
	output[31:0] res,
	output zero, overflow,
	output finish
);

	reg state;
	assign finish = state == 1'b1;
	initial begin
		state = 0;
	end

	reg[3:0] Control;
	reg[31:0] A, B;

	always@(posedge clk) begin
		if(EN & ~state) begin // state == 0
			A <= ALUA;
			B <= ALUB;
			Control <= ALUControl;
			//...to fill sth.in
			state <= 1;
		end
		else state <= 0;
	end
	
	localparam ALU_ADD  = 4'b0001;
	localparam ALU_SUB  = 4'b0010;
	localparam ALU_AND  = 4'b0011;
	localparam ALU_OR   = 4'b0100;
	localparam ALU_XOR  = 4'b0101;
	localparam ALU_SLL  = 4'b0110;
	localparam ALU_SRL  = 4'b0111;
	localparam ALU_SLT  = 4'b1000;
	localparam ALU_SLTU = 4'b1001;
	localparam ALU_SRA  = 4'b1010;
	localparam ALU_Ap4  = 4'b1011;
	localparam ALU_Bout = 4'b1100;

	wire[4:0] shamt = B[4:0];
	wire[32:0] res_subu = {1'b0,A} - {1'b0,B};

	wire[31:0] res_ADD  = A + B;
	wire[31:0] res_SUB  = A - B;
	wire[31:0] res_AND  = A & B;
	wire[31:0] res_OR   = A | B;
	wire[31:0] res_XOR  = A ^ B;
	wire[31:0] res_SLL  = A << shamt;
	wire[31:0] res_SRL  = A >> shamt;

	wire add_of = A[31] & B[31] & ~res_ADD[31] | // neg + neg = pos
					~A[31] & ~B[31] & res_ADD[31]; // pos + pos = neg
	wire sub_of = ~A[31] & B[31] & res_SUB[31] | // pos - neg = neg
					A[31] & ~B[31] & ~res_SUB[31]; // neg - pos = pos
	
	wire[31:0] res_SLT  = {31'b0, res_SUB[31] ^ sub_of};
	wire[31:0] res_SLTU = {31'b0, res_subu[32]};
	wire[31:0] res_SRA  = {{32{A[31]}}, A} >> shamt;
	wire[31:0] res_Ap4  = A + 4;
	wire[31:0] res_Bout = B;

	wire ADD  = Control == ALU_ADD ;
	wire SUB  = Control == ALU_SUB ;
	wire AND  = Control == ALU_AND ;
	wire OR   = Control == ALU_OR  ;
	wire XOR  = Control == ALU_XOR ;
	wire SLL  = Control == ALU_SLL ;
	wire SRL  = Control == ALU_SRL ;
	wire SLT  = Control == ALU_SLT ;
	wire SLTU = Control == ALU_SLTU;
	wire SRA  = Control == ALU_SRA ;
	wire Ap4  = Control == ALU_Ap4 ;
	wire Bout = Control == ALU_Bout;

	
	assign zero = ~|res;
	
	assign overflow = (Control == ALU_ADD && add_of) | 
					(Control == ALU_SUB && sub_of);
	
	assign res = {32{ADD }} & res_ADD  |
					{32{SUB }} & res_SUB  |
					{32{AND }} & res_AND  |
					{32{OR  }} & res_OR   |
					{32{XOR }} & res_XOR  |
					{32{SLL }} & res_SLL  |
					{32{SRL }} & res_SRL  |
					{32{SLT }} & res_SLT  |
					{32{SLTU}} & res_SLTU |
					{32{SRA }} & res_SRA  |
					{32{Ap4 }} & res_Ap4  |
					{32{Bout}} & res_Bout ;

endmodule
\end{lstlisting}

\subparagraph{FU\_mem} FU\_mem模块负责内存访问,需要三个时钟周期完成,同样是通过state与reg寄存器实现内存读写时钟周期的控制。
\begin{lstlisting}[language = {verilog}]
`timescale 1ns / 1ps

module FU_mem(
	input clk, EN, mem_w,
	input[2:0] bhw,
	input[31:0] rs1_data, rs2_data, imm,
	output[31:0] mem_data,
	output finish
);

	reg[1:0] state;
	assign finish = state[0] == 1'b1;
	initial begin
		state = 0;
	end

	reg mem_w_reg;
	reg[2:0] bhw_reg;
	reg[31:0] rs1_data_reg, rs2_data_reg, imm_reg;

	//to fill sth.in
	always @(posedge clk) begin
		if(EN & ~state) begin
			mem_w_reg <= mem_w;
			bhw_reg <= bhw;
			rs1_data_reg <= rs1_data;
			rs2_data_reg <= rs2_data;
			imm_reg <= imm;
			state[1] <= 1'b1;
		end
		else if(~finish) begin
			state = state >> 1;
		end
		else begin
			state <= 0;
		end
	end

	RAM_B ram(.clka(clk),.addra(rs1_data_reg + imm_reg),.dina(rs2_data_reg),.wea(mem_w_reg),
		.douta(mem_data),.mem_u_b_h_w(bhw_reg));

endmodule
\end{lstlisting}

\subparagraph{FU\_div}
FU\_div复杂除法模块,其中除法器由IP核实现,值得注意的是除法器的完成信号是通过state状态与除法器的res\_valid信号共同控制。
\begin{lstlisting}[language = {verilog}]
	`timescale 1ns / 1ps

module FU_div(
    input clk, EN,
    input[31:0] A, B,
    output[31:0] res,
    output finish
);

    wire res_valid;
    wire[63:0] divres;
    
    reg state;
    assign finish = res_valid & state;
    initial begin
        state = 0;
    end

    reg A_valid, B_valid;
    reg[31:0] A_reg, B_reg;


    always @(posedge clk) begin
        if(EN & ~state) begin
            A_valid = ~(A == 0);
            B_valid = ~(B == 0);
            A_reg = A;
            B_reg = B;
            state <= 1;
        end
        else if(res_valid) 
            state <= 0;
        else state <= state;
    end
    //...             //to fill sth.in


    divider div(.aclk(clk),
        .s_axis_dividend_tvalid(A_valid),
        .s_axis_dividend_tdata(A_reg),
        .s_axis_divisor_tvalid(B_valid), 
        .s_axis_divisor_tdata(B_reg),
        .m_axis_dout_tvalid(res_valid), 
        .m_axis_dout_tdata(divres)
    );

    assign res = divres[63:32];

endmodule
\end{lstlisting}

\subparagraph{FU\_jump}
FU\_jump模块负责跳转,需要两个时钟周期完成,同样是通过state与reg寄存器实现跳转时钟周期的控制。
\begin{lstlisting}[language = {verilog}]
`timescale 1ns / 1ps

module FU_jump(
	input clk, EN, JALR,
	input[2:0] cmp_ctrl,
	input[31:0] rs1_data, rs2_data, imm, PC,
	output[31:0] PC_jump, PC_wb,
	output cmp_res, finish
);

	reg state;
	assign finish = state == 1'b1;
	initial begin
		state = 0;
	end

	reg JALR_reg;
	reg[2:0] cmp_ctrl_reg;
	reg[31:0] rs1_data_reg, rs2_data_reg, imm_reg, PC_reg;

	always @(posedge clk) begin
		if(EN & ~state) begin
			JALR_reg <= JALR;
			cmp_ctrl_reg <= cmp_ctrl;
			rs1_data_reg <= rs1_data;
			rs2_data_reg <= rs2_data;
			imm_reg <= imm;
			PC_reg <= PC;
			state <= 1;
		end
		else state <= 0;
	end
	//...            //to fill sth.in
	wire tmp_res;
	cmp_32 cmp(.a(rs1_data_reg), 
				.b(rs2_data_reg), 
				.ctrl(cmp_ctrl_reg), 
				.c(tmp_res)
			);
	
	assign cmp_res = (JALR_reg | cmp_ctrl_reg == 3'b000) ? 1'b1 : tmp_res; 
	assign PC_wb = PC_reg + 4;
	assign PC_jump = JALR_reg ? rs1_data_reg + imm_reg : PC_reg + imm_reg;
endmodule
\end{lstlisting}

\subparagraph{FU\_mul}
FU\_mul阶段需要7个周期完成,通过IP核与state右移实现。
\begin{lstlisting}[language = {verilog}]
	`timescale 1ns / 1ps

module FU_mul(
    input clk, EN,
    input[31:0] A, B,
    output[31:0] res,
    output finish
);

    reg[6:0] state;
    assign finish = state[0] == 1'b1;
    initial begin
        state = 0;
    end

    reg[31:0] A_reg, B_reg;
    wire[63:0] mulres;
    
    //to fill sth.in
    always @(posedge clk) begin
        if (EN & ~state) begin
            A_reg <= A;
            B_reg <= B;
            state[6] <= 1'b1;
        end
        else if(~finish) begin
            state <= state >> 1;
        end
        else state <= 0;
    end
    multiplier mul(.CLK(clk),.A(A_reg),.B(B_reg),.P(mulres));

    assign res = mulres[31:0];

endmodule
\end{lstlisting}