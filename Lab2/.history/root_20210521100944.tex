%! TEX program = xelatex

\documentclass[a4paper,12pt]{article}
\usepackage[UTF8]{ctex}
\input{package.tex}
%! TEX program = xelatex
%! TEX root = root.tex

\setlength{\topmargin}{5pt}
\setlength{\headheight}{5pt}
\setlength{\headsep}{5pt}
\setlength{\footskip}{30pt}
\setlength{\voffset}{-5pt}
\setlength{\hoffset}{16pt}
\setlength{\oddsidemargin}{0pt}
\setlength{\evensidemargin}{\oddsidemargin}
\setlength{\textwidth}{\paperwidth-2\hoffset-2\oddsidemargin-2in+2em}
\setlength{\marginparpush}{0pt}
\setlength{\marginparwidth}{0pt}
\setlength{\parindent}{0em}
\addtolength{\textheight}{4\baselineskip}
\lstset{extendedchars = false}

\newcommand{\coursename}{计算机体系结构}
\newcommand{\studentname}{曾帅 王异鸣}
\newcommand{\collegename}{计算机学院}
\newcommand{\facultyname}{图灵1901}
\newcommand{\majorname}{计算机科学与技术}
\newcommand{\studentid}{3190105729 3190102780}
\newcommand{\teachername}{姜晓红}
\newcommand{\exptype}{实验类型}
\newcommand{\expname}{Pipelined CPU}
\newcommand{\explocation}{实验平台}
\newcommand{\exptime}{实验日期}

\newcommand{\chuhao}{\fontsize{42pt}{\baselineskip}\selectfont}
\newcommand{\xiaochuhao}{\fontsize{36pt}{\baselineskip}\selectfont}
\newcommand{\yihao}{\fontsize{28pt}{\baselineskip}\selectfont}
\newcommand{\erhao}{\fontsize{21pt}{\baselineskip}\selectfont}
\newcommand{\xiaoerhao}{\fontsize{18pt}{\baselineskip}\selectfont}
\newcommand{\sanhao}{\fontsize{15.75pt}{\baselineskip}\selectfont}
\newcommand{\sihao}{\fontsize{14pt}{\baselineskip}\selectfont}
\newcommand{\xiaosihao}{\fontsize{12pt}{\baselineskip}\selectfont}
\newcommand{\wuhao}{\fontsize{10.5pt}{\baselineskip}\selectfont}
\newcommand{\xiaowuhao}{\fontsize{9pt}{\baselineskip}\selectfont}
\newcommand{\liuhao}{\fontsize{7.875pt}{\baselineskip}\selectfont}
\newcommand{\qihao}{\fontsize{5.25pt}{\baselineskip}\selectfont}

\renewcommand{\abstractname}{\sanhao 摘 \quad 要\\}
\renewcommand{\thetable}{\thesection{}.\arabic{table}}
\renewcommand{\thefigure}{\thesection{}.\arabic{figure}}
\renewcommand{\cftsecleader}{\cftdotfill{\cftdotsep}}


\lstset{
    %backgroundcolor=\color{red!50!green!50!blue!50},%代码块背景色为浅灰色
    rulesepcolor= \color{gray}, %代码块边框颜色
    basicstyle = \footnotesize,
    breaklines=true,  %代码过长则换行
    numbers=left, %行号在左侧显示
    numberstyle= \small,%行号字体
    %keywordstyle= \color{blue},%关键字颜色
    commentstyle=\color{gray}, %注释颜色
    frame=shadowbox%用方框框住代码块
}
\begin{document}
\captionsetup{font={footnotesize}}

\input{cover.tex}

% \newpage
% \begin{abstract}
    这里是摘要
\end{abstract}


\newpage
\tableofcontents

\newpage
%%! TEX program = xelatex
%! TEX root = ../root.tex
\section{实验目的}
本次实验要求我们实现加入了cache的CPU流水线
\begin{itemize}
    \item [1.] 理解支持多周期操作流水线的设计原理
    \item [2.] 掌握支持多周期操作流水线的设计方法
    \item [3.] 掌握验证流水线正确性的方法,并根据设计思路进行验证
\end{itemize}    
%! TEX program = xelatex
%! TEX root = ../root.tex

\section{实验内容}
实验的基本要求是实现RISC-V架构下,指令集为RV32I并支持Hazard, forwarding和Predict-not-taken的流水线CPU。 \\
本实验已给出主要框架,需要完成的实验内容如下:
\begin{itemize}
    \item [1.] 补充数据通路
    \item [2.] 设计ByPass Unit
    \item [3.] 设计CPU Controller
    \item [4.] 利用测试程序验证CPU并观察指令的执行情况
\end{itemize}  
%! TEX program = xelatex
%! TEX root = ../root.tex

\section{实验原理}
\subsection{Cache Management Unit}
本实验中需要实现的主要模块就是CMU,即Cache Management Unit,CMU中内含cache模块,其负责处理数据并将其传入或传出cache,并完成cache与CPU和memory的交互。
\begin{figure}[H] %H为当前位置,!htb为忽略美学标准,htbp为浮动图形
	\centering %图片居中
	\includegraphics[width=1.0\textwidth]{figs/1.png} %插入图片,[]中设置图片大小,{}中是图片文件名
	\caption{CMU模块} %最终文档中希望显示的图片标题
	\label{Fig.1} %用于文内引用的标签
\end{figure}
CMU模块安置在流水线的MEM阶段,其与CPU之间有一系列的接口,包括Data\_Read,Data\_Write,Address等,这提供了CPU与cache之间数据传输的通道。同时,CMU还与Memory之间有一系列接口,以用于cache和memory之间进行数据传输。
\subsection{Cache Operation Flow}
CMU中的操作流程如下图所示
\begin{figure}[H] %H为当前位置,!htb为忽略美学标准,htbp为浮动图形
	\centering %图片居中
	\includegraphics[width=1.0\textwidth]{figs/3.png} %插入图片,[]中设置图片大小,{}中是图片文件名
	\caption{Cache Operation Flow} %最终文档中希望显示的图片标题
	\label{Fig.2} %用于文内引用的标签
\end{figure}
当CPU需要对数据进行操作时,会首先传入read或write bit。CMU接收到信号后首先查看地址是否hit,若hit则直接读或写数据。否则检查需要置换的block是否dirty,若dirty则需要先将当前块写回memory。之后从memory中取数据存入cache,再进行读或写操作。
\subsection{Cache Management State Machine}
CMU的内部结构实际上是一个有限状态机,其有S\_IDLE、S\_PRE\_BACK、S\_BACK、S\_FILL、S\_WAIT等5个状态。
\begin{figure}[H] %H为当前位置,!htb为忽略美学标准,htbp为浮动图形
	\centering %图片居中
	\includegraphics[width=1.0\textwidth]{figs/2.png} %插入图片,[]中设置图片大小,{}中是图片文件名
	\caption{CMU状态机} %最终文档中希望显示的图片标题
	\label{Fig.3} %用于文内引用的标签
\end{figure}
cache操作均发生在当前state的下降沿,memory操作均发生在当前state的上升沿。 每个状态的具体描述如下:
\begin{itemize}
	\item [1.] S\_IDLE:空闲状态,不进行memory操作,cache操作hit的情况下一直处于这个状态。
	\item [2.] S\_PRE\_BACK:为了写回,先进行一次读cache。
	\item [3.] S\_BACK:上升沿将上个状态的数据写回到memory,下降沿从cache读下次需要写回的数据,由计数器控制直到整个cache line全部写回。由于memory设置为4个周期完成读写操作,因此需要等待memory给出ack信号,才能进行状态的改变。
	\item [4.] S\_FILL:上升沿从memory读取数据,下降沿向cache写入数据,由计数器控制直到整个cache line全部写入。与S\_BACK类似,需要等待ack信号。
	\item [5.] S\_WAIT:执行之前由于miss而不能进行的cache操作。
\end{itemize}
\subsection{源代码}
\begin{lstlisting}[language = {verilog}]
module cmu (
	// CPU side
	input clk,
	input rst,
	input [31:0] addr_rw,
	input en_r,
	input en_w,
	input [2:0] u_b_h_w,
	input [31:0] data_w,
	output [31:0] data_r,
	output stall,
	
	// mem side
	output reg mem_cs_o = 0,
	output reg mem_we_o = 0,
	output reg [31:0] mem_addr_o = 0,
	input [31:0] mem_data_i,
	output [31:0] mem_data_o,
	input mem_ack_i,
	
	// debug info
	output [2:0] cmu_state
	);
	
	`include "addr_define.vh"
	
	reg [ADDR_BITS-1:0] cache_addr = 0;
	reg cache_load = 0;
	reg cache_store = 0;
	reg cache_edit = 0;
	reg [2:0] cache_u_b_h_w = 0;
	reg [WORD_BITS-1:0] cache_din = 0;
	wire cache_hit;
	wire [WORD_BITS-1:0] cache_dout;
	wire cache_valid;
	wire cache_dirty;
	wire [TAG_BITS-1:0] cache_tag;
	
	cache CACHE (
	.clk(~clk),
	.rst(rst),
	.addr(cache_addr),
	.load(cache_load),
	.store(cache_store),
	.edit(cache_edit),
	.invalid(1'b0),
	.u_b_h_w(cache_u_b_h_w),
	.din(cache_din),
	.hit(cache_hit),
	.dout(cache_dout),
	.valid(cache_valid),
	.dirty(cache_dirty),
	.tag(cache_tag)
	);
	
	localparam
	S_IDLE = 0,
	S_PRE_BACK = 1,
	S_BACK = 2,
	S_FILL = 3,
	S_WAIT = 4;
	
	reg [2:0]state = 0;
	reg [2:0]next_state = 0;
	reg [ELEMENT_WORDS_WIDTH-1:0]word_count = 0;
	reg [ELEMENT_WORDS_WIDTH-1:0]next_word_count = 0;
	assign cmu_state = state;
	
	always @ (posedge clk) begin
	if (rst) begin
	state <= S_IDLE;
	word_count <= 2'b00;
	end
	else begin
	state <= next_state;
	word_count <= next_word_count;
	end
	end
	
	// state ctrl
	always @ (*) begin
	if (rst) begin
	next_state = S_IDLE;
	next_word_count = 2'b00;
	end
	else begin
	case (state)
	S_IDLE: begin
	if (en_r || en_w) begin
	if (cache_hit)
	next_state = S_IDLE;
	else if (cache_valid && cache_dirty)
	next_state = S_PRE_BACK;
	else
	next_state = S_FILL;
	end
	next_word_count = 2'b00;
	end
	
	S_PRE_BACK: begin
	next_state = S_BACK;
	next_word_count = 2'b00;
	end
	
	S_BACK: begin //?
	if (mem_ack_i && word_count == {ELEMENT_WORDS_WIDTH{1'b1}})    // 2'b11 in default case
	next_state = S_FILL;
	else
	next_state = S_BACK;
	
	if (mem_ack_i)
	next_word_count = word_count + 2'b01; //?
	else
	next_word_count = word_count;
	end
	
	S_FILL: begin
	if (mem_ack_i && word_count == {ELEMENT_WORDS_WIDTH{1'b1}})
	next_state = S_WAIT;
	else
	next_state = S_FILL;
	
	if (mem_ack_i)
	next_word_count = word_count + 2'b01;
	else
	next_word_count = word_count;
	end
	
	S_WAIT: begin
	next_state = S_IDLE;
	next_word_count = 2'b00;
	end
	endcase
	end
	end
	
	// cache ctrl
	always @ (*) begin
	case(state)
	S_IDLE, S_WAIT: begin
	cache_addr = addr_rw;
	cache_load = en_r;
	cache_edit = en_w;
	cache_store = 1'b0;
	cache_u_b_h_w = u_b_h_w;
	cache_din = data_w;
	end
	S_BACK, S_PRE_BACK: begin
	cache_addr = {addr_rw[ADDR_BITS-1:BLOCK_WIDTH], next_word_count, {ELEMENT_WORDS_WIDTH{1'b0}}};
	cache_load = 1'b0;
	cache_edit = 1'b0;
	cache_store = 1'b0;
	cache_u_b_h_w = 3'b010;
	cache_din = 32'b0;
	end
	S_FILL: begin
	cache_addr = {addr_rw[ADDR_BITS-1:BLOCK_WIDTH], word_count, {ELEMENT_WORDS_WIDTH{1'b0}}};
	cache_load = 1'b0;
	cache_edit = 1'b0;
	cache_store = mem_ack_i;
	cache_u_b_h_w = 3'b010;
	cache_din = mem_data_i;
	end
	endcase
	end
	assign data_r = cache_dout;
	
	// mem ctrl
	always @ (*) begin
	case (next_state)
	S_IDLE, S_PRE_BACK, S_WAIT: begin
	mem_cs_o = 1'b0;
	mem_we_o = 1'b0;
	mem_addr_o = 32'b0;
	end
	
	S_BACK: begin
	mem_cs_o = 1'b1;
	mem_we_o = 1'b1;
	mem_addr_o = {cache_tag, addr_rw[ADDR_BITS-TAG_BITS-1:BLOCK_WIDTH], next_word_count, {ELEMENT_WORDS_WIDTH{1'b0}}};
	end
	
	S_FILL: begin
	mem_cs_o = 1'b1;
	mem_we_o = 1'b0;
	mem_addr_o = {addr_rw[ADDR_BITS-1:BLOCK_WIDTH], next_word_count, {ELEMENT_WORDS_WIDTH{1'b0}}};
	end
	endcase
	end
	assign mem_data_o = cache_dout;
	
	//important
	assign stall = (next_state != S_IDLE);
	
endmodule
	
\end{lstlisting}
%! TEX program = xelatex
%! TEX root = ../root.tex

\section{实验步骤与调试}
\subsection{仿真} 根据已经写好的代码,进行仿真模拟\\
\subsubsection{异常处理}
当程序遇到exception时,跳入处理程序。程序在WB阶段接收exception信号,并跳入异常处理程序,将还在执行的命令全部清空,并取消寄存器的写入。
\begin{figure}[H] %H为当前位置,!htb为忽略美学标准,htbp为浮动图形
	\centering %图片居中
	\includegraphics[width=1.0\textwidth]{figs/1.png} %插入图片,[]中设置图片大小,{}中是图片文件名
	\caption{仿真结果图1} %最终文档中希望显示的图片标题
	\label{Fig.11} %用于文内引用的标签
\end{figure}
当异常处理程序运行至末尾时,程序读取到mret指令并回到原程序中,由于在异常处理程序中给mepc加上了4,实际回到的是异常发生指令的下一条指令。
\begin{figure}[H] %H为当前位置,!htb为忽略美学标准,htbp为浮动图形
	\centering %图片居中
	\includegraphics[width=1.0\textwidth]{figs/2.png} %插入图片,[]中设置图片大小,{}中是图片文件名
	\caption{仿真结果图2} %最终文档中希望显示的图片标题
	\label{Fig.12} %用于文内引用的标签
\end{figure}
之后程序继续运行直到遇到下一个异常,并以相同的方式处理该异常。
\begin{figure}[H] %H为当前位置,!htb为忽略美学标准,htbp为浮动图形
	\centering %图片居中
	\includegraphics[width=1.0\textwidth]{figs/3.png} %插入图片,[]中设置图片大小,{}中是图片文件名
	\caption{仿真结果图3} %最终文档中希望显示的图片标题
	\label{Fig.13} %用于文内引用的标签
\end{figure}
\subsubsection{中断处理}
当程序遇到外部中断时,进入中断处理程序。在中断过程中,虽然中断信号一直拉起,但是由于mstatus的mie位此时为零,因此无法在中断中触发中断。
\begin{figure}[H] %H为当前位置,!htb为忽略美学标准,htbp为浮动图形
	\centering %图片居中
	\includegraphics[width=1.0\textwidth]{figs/4.png} %插入图片,[]中设置图片大小,{}中是图片文件名
	\caption{仿真结果图4} %最终文档中希望显示的图片标题
	\label{Fig.14} %用于文内引用的标签
\end{figure}
返回时回到需要执行的下一条指令
\begin{figure}[H] %H为当前位置,!htb为忽略美学标准,htbp为浮动图形
	\centering %图片居中
	\includegraphics[width=1.0\textwidth]{figs/5.png} %插入图片,[]中设置图片大小,{}中是图片文件名
	\caption{仿真结果图5} %最终文档中希望显示的图片标题
	\label{Fig.15} %用于文内引用的标签
\end{figure}
返回后中断信号依旧处于拉起状态,在程序完全离开中断处理程序后,再次进入中断处理程序。
\begin{figure}[H] %H为当前位置,!htb为忽略美学标准,htbp为浮动图形
	\centering %图片居中
	\includegraphics[width=1.0\textwidth]{figs/6.png} %插入图片,[]中设置图片大小,{}中是图片文件名
	\caption{仿真结果图6} %最终文档中希望显示的图片标题
	\label{Fig.16} %用于文内引用的标签
\end{figure}
\subsection{综合} 选择左侧面板的Run Synthesis或者点击上方的绿色小三角,选择Synthesis
\subsection{实现} 选择左侧面板的Run Implementation或者点击上方的绿色小三角,选择Implementation。值得注意的是执行implementation之前应该确保引脚约束存在且正确,同时之前已经综合过最新的代码。
\subsection{验证设计} 选择左侧面板的Open Elaborated Design,输出的结果如下,根据原理图来判断,基本没有问题
% \begin{figure}[H] %H为当前位置,!htb为忽略美学标准,htbp为浮动图形
%     \centering %图片居中
%     \includegraphics[width=1.0\textwidth]{yanzhen.png} %插入图片,[]中设置图片大小,{}中是图片文件名
%     \caption{验证结果图} %最终文档中希望显示的图片标题
%     \label{Fig.6} %用于文内引用的标签
% \end{figure}
\subsection{生成二进制文件} 选择左侧面板的Generate Bitstream或者点击上上的绿色二进制标志。同时生成Bitstream前要确保:之前已经综合、实现过最新的代码。如没有,直接运行会默认从综合、实现开始。此过程还要注意生成的bit文件默认存放在.runs下相应的implementation文件夹中
\subsection{烧写上板} 点击左侧的Open Hardware Manager $\rightarrow$ 点击Open Target $\rightarrow$ Auto Connect $\rightarrow$ 点击Program Device $\rightarrow$ 选择bistream路径,烧写。验证结果见实验结果部分。

\section{实验结果与分析}
经过综合实现,我们可以在开发板上看到测试的结果。我们分以下几个部分验证实验结果的正确性

\section{讨论与心得}
本实验要求在上一个实验的基础上实现中断与异常的相关功能。在本次实验中,我们再次复习了上学期与中断部分相关的内容,并学会了如何在使用流水线的情况下进行中断与异常的相关处理。在实验过程中,我们也遇到了一些问题,如中断与异常时处理mepc到底有哪些区别,中断长开时应该如何处理。许多内容在课上都没有提及,我们在查阅了相关资料后得到了答案。
% \newpage
% \input{body/ref.tex}
\end{document}

\begin{thebibliography}{99}  
    \bibitem{ref1}Zheng L, Wang S, Tian L, et al., Query-adaptive late fusion for image search and person re-identification, Proceedings of the IEEE Conference on Computer Vision and Pattern Recognition, 2015: 1741-1750.  
    \bibitem{ref2}Arandjelović R, Zisserman A, Three things everyone should know to improve object retrieval, Computer Vision and Pattern Recognition (CVPR), 2012 IEEE Conference on, IEEE, 2012: 2911-2918.  
    \bibitem{ref3}Lowe D G. Distinctive image features from scale-invariant keypoints, International journal of computer vision, 2004, 60(2): 91-110.  
    \bibitem{ref4}Philbin J, Chum O, Isard M, et al. Lost in quantization: Improving particular object retrieval in large scale image databases, Computer Vision and Pattern Recognition, 2008. CVPR 2008, IEEE Conference on, IEEE, 2008: 1-8.  
\end{thebibliography}