\section{实验结果与分析}
\subsection{Cache仿真结果分析}
为了实现Cache正确性的仿真验证,本次实验采用了对Cache和CMU模块的单独仿真。仿真代码的设计如下:

\begin{lstlisting}[language=Verilog]
    module cmu_sim (
        input wire clk,
        input wire rst,
        output reg [7:0] clk_count = 0,
        output reg [7:0] inst_count = 0,
        output reg [7:0] hit_count = 0
        );
        
        // instruction
        reg [3:0] index = 0;
        wire valid;
        wire write;
        wire [31:0] addr;
        wire [2:0] u_b_h_w;
        wire stall;
        inst INST (
            .clk(clk),
            .rst(rst),
            .index(index),
            .valid(valid),
            .write(write),
            .addr(addr),
            .u_b_h_w(u_b_h_w)
        );
    
        always @(posedge clk) begin
            if (rst)
                index <= 0;
            else if (valid && ~stall)
                index <= index + 1'h1;
        end
        // ram
        wire mem_cs;
        wire mem_we;
        wire [31:0] mem_addr;
        wire [31:0] mem_din;
        wire [31:0] mem_dout;
        wire mem_ack;
         data_ram RAM (
            .clk(clk),
            .rst(rst),
            .addr({21'b0, mem_addr[10:0]}),
            .cs(mem_cs),
            .we(mem_we),
            .din(mem_din),
            .dout(mem_dout),
            .stall(),
            .ack(mem_ack),
            .ram_state()
            );
        
        // cache
        wire [31:0] data_r;
        cmu CMU (
            .clk(clk),
            .rst(rst),
            .addr_rw(addr),
            .u_b_h_w(u_b_h_w),
            .en_r(~write),
            .data_r(data_r),
            .en_w(write),
            .data_w({16'h5678, clk_count, inst_count}),
            .stall(stall),
            .mem_cs_o(mem_cs),
            .mem_we_o(mem_we),
            .mem_addr_o(mem_addr),
            .mem_data_i(mem_dout),
            .mem_data_o(mem_din),
            .mem_ack_i(mem_ack),
            .cmu_state()
        );
    
        // counter
        reg stall_prev;
        
        always @(posedge clk) begin
            if (rst)
                stall_prev <= 0;
            else
                stall_prev <= stall;
        end
        
        always @(posedge clk) begin
            if (rst) begin
                clk_count <= 0;   // 时钟计数
                inst_count <= 0;  // 指令计数
                hit_count <= 0;   // 命中计数
            end
            else if (valid) begin
                clk_count <= clk_count + 1'h1;
                inst_count <= index + 1'h1;
                if (~stall_prev && ~stall)
                    hit_count <= hit_count + 1'h1;
            end
        end
    
    endmodule    
\end{lstlisting}